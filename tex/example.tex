

% %# -*- coding: utf-8-unix -*-
% % !TEX program = xelatex
% % !TEX root = ../thesis.tex
% % !TEX encoding = UTF-8 Unicode
% %%==================================================
% %% chapter02.tex for SJTU Master Thesis
% %% based on CASthesis
% %% modified by wei.jianwen@gmail.com
% %% Encoding: UTF-8
% %%==================================================

% \chapter{{\LaTeX} 排版例子}
% \label{chap:example}

% \section{列表环境}
% \label{sec:list}

% \subsection{无序列表}
% \label{sec:unorderlist}

% 以下是一个无序列表的例子,列表的每个条目单独分段。

% \begin{itemize}
%   \item 这是一个无序列表。
%   \item 这是一个无序列表。
%   \item 这是一个无序列表。
% \end{itemize}

% 使用\verb+itemize*+环境可以创建行内无序列表。
% \begin{itemize*}
%   \item 这是一个无序列表。
%   \item 这是一个无序列表。
%   \item 这是一个无序列表。
% \end{itemize*}
% 行内无序列表条目不单独分段,所有内容直接插入在原文的段落中。

% \subsection{有序列表}
% \label{sec:orderlist}

% 使用环境\verb+enumerate+和\verb+enumerate*+创建有序列表,
% 使用方法无序列表类似。

% \begin{enumerate}
%   \item 这是一个有序列表。
%   \item 这是一个有序列表。
%   \item 这是一个有序列表。
% \end{enumerate}

% 使用\verb+enumerate*+环境可以创建行内有序列表。
% \begin{enumerate*}
%   \item 这是一个默认有序列表。
%   \item 这是一个默认有序列表。
%   \item 这是一个默认有序列表。
% \end{enumerate*}
% 行内有序列表条目不单独分段,所有内容直接插入在原文的段落中。

% \subsection{描述型列表}

% 使用环境\verb+description+可创建带有主题词的列表,条目语法是\verb+\item[主题] 内容+。
% \begin{description}
%     \item[主题一] 详细内容
%     \item[主题二] 详细内容
%     \item[主题三] 详细内容 \ldots
% \end{description}

% \subsection{自定义列表样式}

% 可以使用\verb+label+参数控制列表的样式,
% 详细可以参考WikiBooks\footnote{\url{https://en.wikibooks.org/wiki/LaTeX/List_Structures\#Customizing_lists}}。
% 比如一个自定义样式的行内有序列表
% \begin{enumerate*}[label=\itshape\alph*)\upshape]
%   \item 这是一个自定义样式有序列表。
%   \item 这是一个自定义样式有序列表。
%   \item 这是一个自定义样式有序列表。
% \end{enumerate*}

% \section{数学排版}
% \label{sec:matheq}

% \subsection{公式排版}
% \label{sec:eqformat}

% 这里有举一个长公式排版的例子,来自\href{http://www.tex.ac.uk/tex-archive/info/math/voss/mathmode/Mathmode.pdf}{《Math mode》}:

% \begin {multline}
%   \frac {1}{2}\Delta (f_{ij}f^{ij})=
%   2\left (\sum _{i<j}\chi _{ij}(\sigma _{i}-
%     \sigma _{j}) ^{2}+ f^{ij}\nabla _{j}\nabla _{i}(\Delta f)+\right .\\
%   \left .+\nabla _{k}f_{ij}\nabla ^{k}f^{ij}+
%     f^{ij}f^{k}\left [2\nabla _{i}R_{jk}-
%       \nabla _{k}R_{ij}\right ]\vphantom {\sum _{i<j}}\right )
% \end{multline}

% \subsection{SI单位}

% 使用\verb+siunitx+宏包可以方便地输入SI单位制单位,例如\verb+\SI{5}{\um}+可以得到\SI{5}{\um}。

% \subsubsection{一个四级标题}
% \label{sec:depth4}

% 这是全文唯一的一个四级标题。在这部分中将演示了mathtools宏包中可伸长符号(箭头、等号的例子)的例子。

% \begin{displaymath}
%     A \xleftarrow[n=0]{} B \xrightarrow[LongLongLongLong]{n>0} C
% \end{displaymath}

% \begin{eqnarray}
%   f(x) & \xleftrightarrow[]{A=B}  & B \\
%   & \xleftharpoondown[below]{above} & B \nonumber \\
%   & \xLeftrightarrow[below]{above} & B
% \end{eqnarray}

% 又如:

% \begin{align}
%   \label{eq:none}
%   & I(X_3;X_4)-I(X_3;X_4\mid{}X_1)-I(X_3;X_4\mid{}X_2) \nonumber \\
%   = & [I(X_3;X_4)-I(X_3;X_4\mid{}X_1)]-I(X_3;X_4\mid{}\tilde{X}_2) \\
%   = & I(X_1;X_3;X_4)-I(X_3;X_4\mid{}\tilde{X}_2)
% \end{align}

% \subsection{定理环境}

% 模板中定义了丰富的定理环境
% algo(算法),thm(定理),lem(引理),prop(命题),cor(推论),defn(定义),conj(猜想),exmp(例),rem(注),case(情形),
% bthm(断言定理),blem(断言引理),bprop(断言命题),bcor(断言推论)。
% amsmath还提供了一个proof(证明)的环境。
% 这里举一个“定理”和“证明”的例子。
% \begin{thm}[留数定理]
% \label{thm:res}
%   假设$U$是复平面上的一个单连通开子集,$a_1,\ldots,a_n$是复平面上有限个点,$f$是定义在$U\backslash \{a_1,\ldots,a_n\}$上的全纯函数,
%   如果$\gamma$是一条把$a_1,\ldots,a_n$包围起来的可求长曲线,但不经过任何一个$a_k$,并且其起点与终点重合,那么:

%   \begin{equation}
%     \label{eq:res}
%     \ointop_{\gamma}f(z)\,\mathrm{d}z = 2\uppi\mathbf{i}\sum^n_{k=1}\mathrm{I}(\gamma,a_k)\mathrm{Res}(f,a_k)
%   \end{equation}

%   如果$\gamma$是若尔当曲线,那么$\mathrm{I}(\gamma, a_k)=1$,因此:

%   \begin{equation}
%     \label{eq:resthm}
%     \ointop_{\gamma}f(z)\,\mathrm{d}z = 2\uppi\mathbf{i}\sum^n_{k=1}\mathrm{Res}(f,a_k)
%   \end{equation}

%       % \oint_\gamma f(z)\, dz = 2\pi i \sum_{k=1}^n \mathrm{Res}(f, a_k ).

%   在这里,$\mathrm{Res}(f, a_k)$表示$f$在点$a_k$的留数,$\mathrm{I}(\gamma,a_k)$表示$\gamma$关于点$a_k$的卷绕数。
%   卷绕数是一个整数,它描述了曲线$\gamma$绕过点$a_k$的次数。如果$\gamma$依逆时针方向绕着$a_k$移动,卷绕数就是一个正数,
%   如果$\gamma$根本不绕过$a_k$,卷绕数就是零。

%   定理\ref{thm:res}的证明。

%   \begin{proof}
%     首先,由……

%     其次,……

%     所以……
%   \end{proof}
% \end{thm}

% 上面的公式例子中,有一些细节希望大家注意。微分号d应该使用“直立体”也就是用mathrm包围起来。
% 并且,微分号和被积函数之间应该有一段小间隔,可以插入\verb+\,+得到。
% 斜体的$d$通常只作为一般变量。
% i,j作为虚数单位时,也应该使用“直立体”为了明显,还加上了粗体,例如\verb+\mathbf{i}+。斜体$i,j$通常用作表示“序号”。
% 其他字母在表示常量时,也推荐使用“直立体”譬如,圆周率$\uppi$(需要upgreek宏包),自然对数的底$\mathrm{e}$。
% 不过,我个人觉得斜体的$e$和$\pi$很潇洒,在不至于引起混淆的情况下,我也用这两个字母的斜体表示对应的常量。


% \section{向文档中插入图像}
% \label{sec:insertimage}

% \subsection{支持的图片格式}
% \label{sec:imageformat}

% \XeTeX 可以很方便地插入PDF、PNG、JPG格式的图片。

% 插入PNG/JPG的例子如\ref{fig:SRR}所示。
% 这两个水平并列放置的图共享一个“图标题”(table caption),没有各自的小标题。

% \begin{figure}[!htp]
%   \centering
%   \includegraphics[width=4cm]{example/sjtulogo.png}
%   \hspace{1cm}
%   \includegraphics[width=4cm]{example/sjtulogo.jpg}
%   \bicaption[这里将出现在插图索引中]
%     {中文题图}
%     {English caption}
%   \label{fig:SRR}
% \end{figure}

% 这里还有插入EPS图像和PDF图像的例子,如图\ref{fig:epspdf:a}和图\ref{fig:epspdf:b}。这里将EPS和PDF图片作为子图插入,每个子图有自己的小标题。子图标题使用subcaption宏包添加。

% \begin{figure}[!htp]
%   \centering
%   \subcaptionbox{EPS 图像\label{fig:epspdf:a}}[3cm] %标题的长度,超过则会换行,如下一个小图。
%     {\includegraphics[height=2.5cm]{example/sjtulogo.eps}}
%   \hspace{4em}
%   \subcaptionbox{PDF 图像,注意这个图略矮些。如果标题很长的话,它会自动换行\label{fig:epspdf:b}}
%     {\includegraphics[height=2cm]{sjtulogo.pdf}}
%   \bicaption{插入eps和pdf的例子(使用 subcaptionbox 方式)}{An EPS and PDF demo with subcaptionbox}
%   \label{fig:pdfeps-subcaptionbox}
% \end{figure}

% \begin{figure}[!htp]
%   \centering
%   \begin{subfigure}{2.5cm}
%     \centering
%     \includegraphics[height=2.5cm]{example/sjtulogo.eps}
%     \caption{EPS 图像}
%   \end{subfigure}
%   \hspace{4em}
%   \begin{subfigure}{0.4\textwidth}
%     \centering
%     \includegraphics[height=2cm]{sjtulogo.pdf}
%     \caption{PDF 图像,注意这个图略矮些。subfigure中同一行的子图在顶端对齐。}
%   \end{subfigure}
%   \bicaption{插入eps和pdf的例子(使用 subfigure 方式)}{An EPS and PDF demo with subfigure}
%   \label{fig:pdfeps-subfigure}
% \end{figure}

% 更多关于 \LaTeX 插图的例子可以参考\href{http://www.cs.duke.edu/junhu/Graphics3.pdf}{《\LaTeX 插图指南》}。

% \subsection{长标题的换行}
% \label{sec:longcaption}

% 图\ref{fig:longcaptionbad}和图\ref{fig:longcaptiongood}都有比较长图标题,通过对比发现,图\ref{fig:longcaptiongood}的换行效果更好一些。
% 其中使用了minipage环境来限制整个浮动体的宽度。

% \begin{figure}[!htp]
%   \centering
%   \includegraphics[width=4cm]{sjtubadge.pdf}
%   \bicaption[这里将出现在插图索引]
%     {上海交通大学是我国历史最悠久的高等学府之一,是教育部直属、教育部与上海市共建的全国重点大学.}
%     {Where there is a will, there is a way.}
%  \label{fig:longcaptionbad}
% \end{figure}

% \begin{figure}[!htbp]
%   \centering
%   \begin{minipage}[b]{0.6\textwidth}
%     \centering
%     \includegraphics[width=4cm]{sjtubadge.pdf}
%     \bicaption[出现在插图索引中]
%       {上海交通大学是我国历史最悠久的高等学府之一,是教育部直属、教育部与上海市共建的全国重点大学.}
%       {Where there is a will, there is a way.}
%     \label{fig:longcaptiongood}
%   \end{minipage}
% \end{figure}

% \subsection{添加图注}

% 当插图中组成部件由数字或字母等编号表示时,可在插图下方添加图注进行说明,如图\ref{fig:cn_100t}所示。

% \begin{figure}[!htp]
%   \centering
%   \includegraphics[width=0.3\textwidth]{example/cn_100t.png}\
%   \begin{center}
%     \small\kaishu 1.立柱 2.提升释放机构 3.标准冲击加速度计 \\ 4.导轨 5.重锤 6.被校力传感器 7.底座
%   \end{center}
%   \vspace{-1em}
%   \bicaption[出现在插图索引中]
%     {示例图片来源于\parencite{he1999}}
%     {Stay hungry, stay foolish.}
%  \label{fig:cn_100t}
% \end{figure}

% \subsection{绘制流程图}

% 图\ref{fig:flow_chart}是一张流程图示意。使用tikz环境,搭配四种预定义节点(\verb+startstop+、\verb+process+、\verb+decision+和\verb+io+),可以容易地绘制出流程图。
% \begin{figure}[!htp]
%     \centering
%     \resizebox{6cm}{!}{\input{figure/example/flow_chart.tex}}
%     \bicaption{绘制流程图效果}{Flow chart}
%     \label{fig:flow_chart}
% \end{figure}

% \clearpage

% \section{表格}
% \label{sec:tab}

% 这一节给出的是一些表格的例子,如表\ref{tab:firstone}所示。

% \begin{table}[!hpb]
%   \centering
%   \bicaption[指向一个表格的表目录索引]
%     {一个颇为标准的三线表格\footnotemark[1]}
%     {A Table}
%   \label{tab:firstone}
%   \begin{tabular}{@{}llr@{}} \toprule
%     \multicolumn{2}{c}{Item} \\ \cmidrule(r){1-2}
%     Animal & Description & Price (\$)\\ \midrule
%     Gnat & per gram & 13.65 \\
%     & each & 0.01 \\
%     Gnu & stuffed & 92.50 \\
%     Emu & stuffed & 33.33 \\
%     Armadillo & frozen & 8.99 \\ \bottomrule
%   \end{tabular}
% \end{table}
% \footnotetext[1]{这个例子来自\href{http://www.ctan.org/tex-archive/macros/latex/contrib/booktabs/booktabs.pdf}{《Publication quality tables in LATEX》}(booktabs宏包的文档)。这也是一个在表格中使用脚注的例子,请留意与threeparttable实现的效果有何不同。}

% 下面一个是一个更复杂的表格,用threeparttable实现带有脚注的表格,如表\ref{tab:footnote}。

% \begin{table}[!htpb]
%   \bicaption[出现在表目录的标题]
%     {一个带有脚注的表格的例子}
%     {A Table with footnotes}
%   \label{tab:footnote}
%   \centering
%   \begin{threeparttable}[b]
%      \begin{tabular}{ccd{4}cccc}
%       \toprule
%       \multirow{2}{6mm}{total}&\multicolumn{2}{c}{20\tnote{1}} & \multicolumn{2}{c}{40} &  \multicolumn{2}{c}{60}\\
%       \cmidrule(lr){2-3}\cmidrule(lr){4-5}\cmidrule(lr){6-7}
%       &www & \multicolumn{1}{c}{k} & www & k & www & k \\ % 使用说明符 d 的列会自动进入数学模式,使用 \multicolumn 对文字表头做特殊处理
%       \midrule
%       &$\underset{(2.12)}{4.22}$ & 120.0140\tnote{2} & 333.15 & 0.0411 & 444.99 & 0.1387 \\
%       &168.6123 & 10.86 & 255.37 & 0.0353 & 376.14 & 0.1058 \\
%       &6.761    & 0.007 & 235.37 & 0.0267 & 348.66 & 0.1010 \\
%       \bottomrule
%     \end{tabular}
%     \begin{tablenotes}
%     \item [1] the first note.% or \item [a]
%     \item [2] the second note.% or \item [b]
%     \end{tablenotes}
%   \end{threeparttable}
% \end{table}

% \section{参考文献管理}

%  \LaTeX 具有将参考文献内容和表现形式分开管理的能力,涉及三个要素:参考文献数据库、参考文献引用格式、在正文中引用参考文献。
% 这样的流程需要多次编译:

% \begin{enumerate}[noitemsep,topsep=0pt,parsep=0pt,partopsep=0pt]
% 	\item 用户将论文中需要引用的参考文献条目,录入纯文本数据库文件(bib文件)。
% 	\item 调用xelatex对论文模板做第一次编译,扫描文中引用的参考文献,生成参考文献入口文件(aux)文件。
% 	\item 调用bibtex,以参考文献格式和入口文件为输入,生成格式化以后的参考文献条目文件(bib)。
% 	\item 再次调用xelatex编译模板,将格式化以后的参考文献条目插入正文。
% \end{enumerate}

% 参考文献数据库(thesis.bib)的条目,可以从Google Scholar搜索引擎\footnote{\url{https://scholar.google.com}}、CiteSeerX搜索引擎\footnote{\url{http://citeseerx.ist.psu.edu}}中查找,文献管理软件Papers\footnote{\url{http://papersapp.com}}、Mendeley\footnote{\url{http://www.mendeley.com}}、JabRef\footnote{\url{http://jabref.sourceforge.net}}也能够输出条目信息。

% 下面是在Google Scholar上搜索到的一条文献信息,格式是纯文本:

% \begin{lstlisting}[caption={从Google Scholar找到的参考文献条目}, label=googlescholar, escapeinside="", numbers=none]
%     @phdthesis{"白2008信用风险传染模型和信用衍生品的定价",
%       title={"信用风险传染模型和信用衍生品的定价"},
%       author={"白云芬"},
%       year={2008},
%       school={"上海交通大学"}
%     }
% \end{lstlisting}

% 推荐修改后在bib文件中的内容为:

% \begin{lstlisting}[caption={修改后的参考文献条目}, label=itemok, escapeinside="", numbers=none]
%   @phdthesis{bai2008,
%     title={"信用风险传染模型和信用衍生品的定价"},
%     author={"白云芬"},
%     date={2008},
%     address={"上海"},
%     school={"上海交通大学"}
%   }
% \end{lstlisting}

% 按照教务处的要求,参考文献外观应符合国标GBT7714的要求\footnote{\url{http://www.cces.net.cn/guild/sites/tmxb/Files/19798_2.pdf}}。
% 在模板中,表现形式的控制逻辑通过biblatex-gb7714-2015包实现\footnote{\url{https://www.ctan.org/pkg/biblatex-gb7714-2015}},基于{Bib\LaTeX}管理文献。在目前的多数TeX发行版中,可能都没有默认包含biblatex-gb7714-2015,需要手动安装。

% 正文中引用参考文献时,用\verb+\cite{key1,key2,key3...}+可以产生“上标引用的参考文献”,
% 如\cite{Meta_CN,chen2007act,DPMG}。
% 使用\verb+\parencite{key1,key2,key3...}+则可以产生水平引用的参考文献,例如\parencite{JohnD,zhubajie,IEEE-1363}。
% 请看下面的例子,将会穿插使用水平的和上标的参考文献:关于书的\parencite{Meta_CN,JohnD,IEEE-1363},关于期刊的\cite{chen2007act,chen2007ewi},
% 会议论文\parencite{DPMG,kocher99,cnproceed},
% 硕士学位论文\parencite{zhubajie,metamori2004},博士学位论文\cite{shaheshang,FistSystem01,bai2008},标准文件\parencite{IEEE-1363},技术报告\cite{NPB2},电子文献\parencite{xiaoyu2001, CHRISTINE1998},用户手册\parencite{RManual}。

% 总结一些注意事项:
% \begin{itemize}
% \item 参考文献只有在正文中被引用了,才会在最后的参考文献列表中出现;
% \item 参考文献“数据库文件”bib是纯文本文件,请使用UTF-8编码,不要使用GBK编码;
% \item 参考文献条目中默认通过date域输入时间。兼容使用year域时会产生编译warning,可忽略。
% \end{itemize}

% \section{用listings插入源代码}

% 原先ctexbook文档类和listings宏包配合使用时,代码在换页时会出现莫名其妙的错误,后来经高人指点,顺利解决了。
% 感兴趣的话,可以看看\href{http://bbs.ctex.org/viewthread.php?tid=53451}{这里}。
% 这里给使用listings宏包插入源代码的例子,这里是一段C代码。
% 另外,listings宏包真可谓博大精深,可以实现各种复杂、漂亮的效果,想要进一步学习的同学,可以参考
% \href{http://mirror.ctan.org/macros/latex/contrib/listings/listings.pdf}{listings宏包手册}。

% \begin{lstlisting}[language={C}, caption={一段C源代码}]
% #include <stdio.h>
% #include <unistd.h>
% #include <sys/types.h>
% #include <sys/wait.h>

% int main() {
%   pid_t pid;

%   switch ((pid = fork())) {
%   case -1:
%     printf("fork failed\n");
%     break;
%   case 0:
%     /* child calls exec */
%     execl("/bin/ls", "ls", "-l", (char*)0);
%     printf("execl failed\n");
%     break;
%   default:
%     /* parent uses wait to suspend execution until child finishes */
%     wait((int*)0);
%     printf("is completed\n");
%     break;
%   }

%   return 0;
% }
% \end{lstlisting}

% \section{用algorithm和algorithmicx宏包插入算法描述}

% algorithmicx 比 algorithmic 增加了一些命令。
% 示例如算法\ref{algo:sum_100}和算法\ref{algo:merge_sort},
% 后者的代码来自\href{http://hustsxh.is-programmer.com/posts/38801.html}{xhSong的博客}。
% algorithmicx的详细使用方法见\href{http://mirror.hust.edu.cn/CTAN/macros/latex/contrib/algorithmicx/algorithmicx.pdf}{官方README}。
% 使用算法宏包时,算法出现的位置很多时候不按照tex文件里的书写顺序,
% 需要强制定位时可以使用\verb+\begin{algorithm}[H]+
% \footnote{http://tex.stackexchange.com/questions/165021/fixing-the-location-of-the-appearance-in-algorithmicx-environment}

% 这是写在算法\ref{algo:sum_100}前面的一段话,在生成的文件里它会出现在算法\ref{algo:sum_100}前面。

% \begin{algorithm}
% % \begin{algorithm}[H] % 强制定位
% \caption{求100以内的整数和}
% \label{algo:sum_100}
% \begin{algorithmic}[1] %每行显示行号
% \Ensure 100以内的整数和 % 输出
% \State $sum \gets 0$
% \For{$i = 0 \to 100$}
%     \State $sum \gets sum + i$
%   \EndFor
% \end{algorithmic}
% \end{algorithm}

% 这是写在两个算法中间的一段话,当算法\ref{algo:sum_100}不使用\verb+\begin{algorithm}[H]+时它也会出现在算法\ref{algo:sum_100}前面。

% 对于很长的算法,单一的算法块\verb+\begin{algorithm}...\end{algorithm}+是不能自动跨页的
% \footnote{http://tex.stackexchange.com/questions/70733/latex-algorithm-not-display-under-correct-section},
% 会出现的情况有:

% \begin{itemize}
%   \item 该页放不下当前的算法,留下大片空白,算法在下一页显示
%   \item 单一页面放不下当前的算法,显示时超过页码的位置直到超出整个页面范围
% \end{itemize}

% 解决方法有:

% \begin{itemize}
%   \item (推荐)使用\verb+algstore{algname}+和\verb+algrestore{algname}+来讲算法分为两个部分\footnote{http://tex.stackexchange.com/questions/29816/algorithm-over-2-pages},如算法\ref{algo:merge_sort}。
%   \item 人工拆分算法为多个小的部分。
% \end{itemize}

% \begin{algorithm}
% % \begin{algorithm}[H] % 强制定位
% \caption{用归并排序求逆序数}
% \label{algo:merge_sort}
% \begin{algorithmic}[1] %每行显示行号
% \Require $Array$数组,$n$数组大小 % 输入
% \Ensure 逆序数 % 输出
% \Function {MergerSort}{$Array, left, right$}
%   \State $result \gets 0$
%   \If {$left < right$}
%     \State $middle \gets (left + right) / 2$
%     \State $result \gets result +$ \Call{MergerSort}{$Array, left, middle$}
%     \State $result \gets result +$ \Call{MergerSort}{$Array, middle, right$}
%     \State $result \gets result +$ \Call{Merger}{$Array,left,middle,right$}
%   \EndIf
%   \State \Return{$result$}
% \EndFunction
% \State %空一行
% \Function{Merger}{$Array, left, middle, right$}
%   \State $i\gets left$
%   \State $j\gets middle$
%   \State $k\gets 0$
%   \State $result \gets 0$
%   \While{$i<middle$ \textbf{and} $j<right$}
%     \If{$Array[i]<Array[j]$}
%       \State $B[k++]\gets Array[i++]$
%     \Else
%       \State $B[k++] \gets Array[j++]$
%       \State $result \gets result + (middle - i)$
%     \EndIf
%   \EndWhile
%   \algstore{MergeSort}
% \end{algorithmic}
% \end{algorithm}

% \begin{algorithm}
% \begin{algorithmic}[1]
%   \algrestore{MergeSort}
%   \While{$i<middle$}
%     \State $B[k++] \gets Array[i++]$
%   \EndWhile
%   \While{$j<right$}
%     \State $B[k++] \gets Array[j++]$
%   \EndWhile
%   \For{$i = 0 \to k-1$}
%     \State $Array[left + i] \gets B[i]$
%   \EndFor
%   \State \Return{$result$}
% \EndFunction
% \end{algorithmic}
% \end{algorithm}

% 这是写在算法\ref{algo:merge_sort}后面的一段话,
% 但是当算法\ref{algo:merge_sort}不使用\verb+\begin{algorithm}[H]+时它会出现在算法\ref{algo:merge_sort}
% 甚至算法\ref{algo:sum_100}前面。

% 对于算法的索引要注意\verb+\caption+和\verb+\label+的位置,
% 必须是先\verb+\caption+再\verb+\label+\footnote{http://tex.stackexchange.com/questions/65993/algorithm-numbering},
% 否则会出现\verb+\ref{algo:sum_100}+生成的编号跟对应算法上显示不一致的问题。

% 根据Werner的回答\footnote{http://tex.stackexchange.com/questions/53357/switch-cases-in-algorithmic}
% 增加了\verb+Switch+和\verb+Case+的支持,见算法\ref{algo:switch_example}。

% \begin{algorithm}
% \caption{Switch示例}
% \label{algo:switch_example}
% \begin{algorithmic}[1]
%   \Switch{$s$}
%     \Case{$a$}
%       \Assert{0}
%     \EndCase
%     \Case{$b$}
%       \Assert{1}
%     \EndCase
%     \Default
%       \Assert{2}
%     \EndDefault
%   \EndSwitch
% \end{algorithmic}
% \end{algorithm}
