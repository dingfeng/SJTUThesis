%# -*- coding: utf-8-unix -*-
% !TEX program = xelatex
% !TEX root = ../thesis.tex
% !TEX encoding = UTF-8 Unicode
%%==================================================
%% abstract.tex for SJTU Master Thesis
%%==================================================

\begin{abstract}

随着信息技术的兴起和应用,许多计算机应用需要用户进行登录操作来身份验证,如密码登录等。为了提升用户体验和信息的安全性,大量基于人体特有生物特征的身份认证技术被不断研究并得到广泛应用,例如指纹识别、人脸识别等。手写签名作为个体的一种重要的行为特征,在金融、法律、政府等领域广泛应用于用户身份的识别,具有极其悠久的历史,然而一个仿造签名可能会造成巨大损失甚至动荡,因此手写签名的正确甄别显得尤为重要。直至目前,手写签名识别主要依靠人工进行,签名验证者根据用户的历史签名来判断当前的签名是否为真实签名,该方式具有人力成本高、依赖验证者经验和误判率高等缺点。由于自动手写签名认证顺应用户对用户体验和安全提升上的诉求,将人力资源从签名识别中解放出来,提高签名识别的准确率和稳定性,因而关于自动签名认证系统的研究和应用逐步开展起来。

目前国内外对手写签名认证的研究大多基于开放数据集,这些研究工作在普适性和用户体验上有待提高。另一方面,探索不同的签名捕捉方式,可以降低签名认证的成本,让签名认证系统在现实中快速普及。本文针对现有研究存在的问题,使用声波记录签名信息,参考传统的签名认证框架,探索出一种新的手写签名认证方案。本文的主要研究成果包括:

(1)提出利用声波感知技术来记录签名运动。使用主流智能手机上扬声器和麦克风进行声波的发射和接收操作,针对商用硬件的特性设计用户友好的声波信号,采用声波相位相关信息实现了对手写签名运动中微小动作的跟踪。

(2)针对声波相位相关信息设计了合适的特征提取和用户独立的识别模型。去除相位相关信息中的环境噪声,将手写签名动作的信息提取出来,结合主流的自动化签名认证系统框架和分类技术,设计出适合声波的手写签名判别流程和模型。

(3)进行实验评估和原型系统实现。对本文方案进行了全面的评估,包括验证精度、交叉用户可用性、硬件补偿的效果、系统鲁棒性、微基准实
验、对比实验和重放攻击等。设计并实现了基于声波的签名认证原型系统,对其运行效率进行评估,该方案系统准确度性能达到:AUC(ROC曲线下方面积)=98.7\% 和EER(等错误率)=5.5\%,实验结果表明该方案是个非侵入式、鲁棒的、安全的、低延迟的手写签名认证方案。

本文首先介绍手写签名认证系统的背景和研究意义,分析当前研究的现状和不足,然后对手写签名认证系统进行简要综述,提出本文的技术路线。在此基础上研究并设计了一种包含四个模块的基于声波的手写签名认证方案,对声波相位相关信号感知技术、特征提取技术、相似性度量技术和深度学习模型等关键技术进行了深入研究,详细介绍了方案中各模块的设计思想和技术细节。接着对该方案设计了多组实验进行评估,设计并实现了一个原型系统。最后总结本文研究内容并展望未来的研究工作。

\end{abstract}

\begin{englishabstract}

With the rise and application of information technology, many computer applications require users to perform login operations to authenticate, such as password login. In order to improve user experience and information security, a large number of identity authentication technologies based on human-specific biometrics have been continuously researched and widely used, such as fingerprint recognition and face recognition. Handwritten signatures, as an important behavioral feature of individuals, have been widely used in the identification of user identities in the fields of finance, law, government, etc., and have a long history. However, a forged signature may cause huge losses and even turbulence. Correct screening is particularly important. Until now, handwritten signature recognition has mainly been performed manually, and the signature verifier determines whether the current signature is a real signature based on the user's historical signature. This method has the disadvantages of high labor costs, reliance on verifier experience, and high rate of false positives. Since the automatic handwritten signature authentication complies with users' demands for user experience and security improvement, it frees human resources from signature recognition and improves the accuracy and stability of signature recognition. Therefore, the research and application of automatic signature authentication systems have been gradually carried out.

At present, researches on handwritten signature authentication at home and abroad are mostly based on open datasets. These researches need to be improved in terms of universality and user experience. On the other hand, exploring different signature capture methods can reduce the cost of signature authentication and make signature authentication systems popular in practice. Aiming at the problems existing in the existing research, this paper uses sonic to record signature information and refers to the traditional signature authentication framework to explore a new handwritten signature authentication scheme. The main research results of this article include:

(1) It is proposed to use sonic sensing technology to record signature movement. Use the speakers and microphones on mainstream smartphones for sound wave transmission and reception operations, design user-friendly sound wave signals for the characteristics of commercial hardware, and use sound wave phase-related information to track small movements in handwritten signature movements.

(2) A suitable feature extraction and user-independent recognition model is designed for the acoustic wave phase related information. The environmental noise in the phase-related information is removed, and the information of the handwritten signature action is extracted. Combining with the mainstream automated signature authentication system framework and classification technology, a handwritten signature discrimination process and model suitable for acoustic waves are designed.

(3) Perform experimental evaluation and prototype system implementation. Comprehensive evaluation of the scheme in this paper, including verification accuracy, cross-user availability, effects of hardware compensation, system robustness, micro-benchmarking
Experiments, comparative experiments, and replay attacks. Designed and implemented a sonic-based signature authentication prototype system and evaluated its operating efficiency. The system's accuracy performance reached: AUC (area under the ROC curve) = 98.7\% and EER (equal error rate) = 5.5\%. Experimental results shows that the scheme is a non-intrusive, robust, secure, low-latency handwritten signature authentication scheme.

This article first introduces the background and research significance of the handwritten signature authentication system, analyzes the current status and shortcomings of the current research, and then briefly summarizes the handwritten signature authentication system and proposes the technical route of this article. Based on this, a sonic-based handwritten signature authentication scheme with four modules is researched and designed, and key technologies such as sonic phase-dependent signal sensing technology, feature extraction technology, similarity measurement technology, and deep learning models are studied in depth. The design ideas and technical details of each module in the scheme are introduced in detail. Then, several schemes of experiments were designed to evaluate the scheme, and a prototype system was designed and implemented. Finally, this article summarizes the research content and looks forward to future research work.

\end{englishabstract}

