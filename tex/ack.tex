%# -*- coding: utf-8-unix -*-
% !TEX program = xelatex
% !TEX root = ../thesis.tex
% !TEX encoding = UTF-8 Unicode
\begin{thanks}
岁月如梭,一眨眼,两年半时间已然过去,仿佛昨天我刚入学,如今却已经开始要毕业走上社会。在此,我想感谢在这珍贵的研究生生涯帮助过我,给我力量的老师、同学和亲人们。是你们的支持和鼓励,让我的这两年半的路途走得如此顺畅。

感谢我的导师王东老师。还记得,第一次见王老师是在夏令营的时候,王老师激情四射的演讲风格深深地吸引了我和我的同学。进入实验室后,除了生活上的关心,王老师给了很多科研上的指导,不断督促我的科研进展,遇到困难时召集师兄们帮我一起解决困难和指引正确的方向,给予适当的鼓励。王老师,除了是科研上的导师外,还是一位人生导师,严谨的处事风格和友善的待人风格对我启发颇深。王老师,是我科研上的导师,也是我饭桌上的朋友。

感谢两位博士师兄——赵润和张谦,两位师兄以丰富的学识征服了我。在组会上,经常能给我的研究给出宝贵的建议,以自己深刻见解指引我走向正确的方向。和两位师兄的接触,也让我虽然不读博也能了解博士的生活,增长了我的见识。赵润师兄是个二次元,也是我附近最年长的师兄,可以给我展示一个不一样的世界,在生活给我提供了很多经验之谈,让我受益匪浅。张谦师兄是个暖男,很友善,易于相处,总能细心解答我的疑问。另外,感谢李冬师兄在我初进实验室时,给予热心的生活和科研上的帮助。和三位师兄一起的澳门之行,让我至今记忆犹新。

感谢实验室和我同届的邓毓峰、陈渤、黄安娜、徐华韬、朱艳、江浩、黄思和秦培杰,是你们在我狭小的研究生圈子里,提供了一个良好健康的生活和学习氛围。实验室同学认真学习和放开玩耍的样子都很酷。我会好好珍惜你们在一起度过的时光,包括舟山春游、物联网志愿者、组会、饭局等,一起玩耍大大增进了我们之间的情谊。

感谢我的父母和家人,是你们的支持使我顺利完成学业,你们的支持和关怀不仅给与我心灵上的蕴藉,还成为我坚持奋斗的不竭动力,是你们让我更加乐观地面对生活中的一切,感谢你们对我如此无私的付出,我将用一生去回报。

再次感谢所有给予我关心、支持和帮助的老师、同学、亲人们!未来的生活和工作中我会继续努力,争取创造更大的成就。

\end{thanks}
